% ------------------------------------------------------------------
%
%  Verwendete Packages und Makro-Dateien
%
% ------------------------------------------------------------------

% ---------------------------------------------------------------------
% Seitenlayout
% ---------------------------------------------------------------------
\usepackage[a4paper,
            centering,
            textwidth=17cm,
            textheight=24cm,
            headheight=14.5pt]{geometry}


% ---------------------------------------------------------------------
% Absatzabst�nde
% ---------------------------------------------------------------------
\parindent0mm
\parskip1ex


% ---------------------------------------------------------------------
% Kopf- und Fu�zeile
% ---------------------------------------------------------------------
\usepackage{fancyhdr}
\fancyhead[OR,EL]{\thepage}
\fancyhead[OL]{\sl\nouppercase\rightmark}
\fancyhead[ER]{\sl\nouppercase\leftmark}
\fancyfoot{}
\pagestyle{fancy}
\renewcommand{\chaptermark}[1]{\markboth{\thechapter~~#1}{}}
\renewcommand{\sectionmark}[1]{\markright{\thesection~~#1}}
\fancypagestyle{plain}{%
  \fancyhead{}
  \fancyfoot{}
  \renewcommand{\headrulewidth}{0pt}
}
\raggedbottom

% ---------------------------------------------------------------------
%  Allgemeine Packages
% ---------------------------------------------------------------------
\usepackage{lmodern}
\usepackage[utf8]{inputenc} 
\usepackage[T1]{fontenc} 
\usepackage[ngerman]{babel}
\usepackage{babelbib}
\usepackage{makeidx}
\usepackage{amsmath}
\usepackage{amsbsy}
\usepackage{amsthm}
\usepackage{amssymb}
\usepackage{amsfonts}
\usepackage{color}
\usepackage{ifpdf}
\usepackage{graphicx}
\usepackage{pdfpages}

		\usepackage[
		nonumberlist, %keine Seitenzahlen anzeigen
		acronym,      %ein Abk�rzungsverzeichnis erstellen
		toc,          %Eintr�ge im Inhaltsverzeichnis
		section]      %im Inhaltsverzeichnis auf section-Ebene erscheinen
		{glossaries} 
		
		\newglossary[slg]{symbolslist}{syi}{syg}{Symbolverzeichnis}
		
		%Den Punkt am Ende jeder Beschreibung deaktivieren
		\renewcommand*{\glspostdescription}{}
		
		%Glossar-Befehle anschalten
		\makeglossaries
		
		%Diese Befehle sortieren die Eintr�ge in den
		%einzelnen Listen:
		%makeindex -s datei.ist -t datei.alg -o datei.acr datei.acn
		%makeindex -s datei.ist -t datei.glg -o datei.gls datei.glo
		%makeindex -s datei.ist -t datei.slg -o datei.syi datei.syg
		
		%Befehle f�r Symbole
		
		\newglossaryentry{symb:Pi}{
		name=$\pi$,
		description={Die Kreiszahl.},
		sort=symbolpi, type=symbolslist
		} 


\ifpdf
  \DeclareGraphicsExtensions{.pdf,.jpg,.png,.ps,.eps}
\else
  \DeclareGraphicsExtensions{.eps,.ps,.png,.jpg,.pdf}
\fi


% ---------------------------------------------------------------------
%  Hyperref-Paket
% ---------------------------------------------------------------------
\ifpdf
  \usepackage{hyperref}
\else
  \usepackage[hypertex]{hyperref}
\fi
\definecolor{dgreen}{rgb}{0,0.7,0}
\definecolor{dblue}{rgb}{0,0,0.7}
\definecolor{dred}{rgb}{0.7,0,0}
\hypersetup{bookmarksnumbered=true,
            plainpages=false,
            colorlinks=true,
            linkcolor=dblue,
            citecolor=dgreen,
            filecolor=dred,
            urlcolor=blue}


